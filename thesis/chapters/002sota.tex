To build a MIP formulation that solves the delete-relaxed task $\Pi^+$, we identify two sections:
\begin{enumerate}
    \item Base IP model: defining variables for atoms, actions and \textit{first achievers} \cite{Imai_15}, aswell as a basic set of constraints, aiming to enforce the basic functionalities of actions;
    \item Modeling acyclicity: ensuring that the solution is in fact a plan, by adding constraints that make sure that a timeline of actions can be constructed from the solution obtained by the MIP solver.
\end{enumerate}
The straightforward IP models, proposed by the literature, are often intractable and not useful in practice for computing $\hplus$; this has lead the research community to develop preprocessing techniques to reduce the size and complexity of the IP models.

\section{Base IP model}
Different models have been proposed in past years; we will use the base model proposed in \cite{Rankooh_22}:\\
$x_a\in\{0,1\}$ indicates whether the action $a$ is used in a solution $\pi$.
$$x_a=
\begin{cases}
    0\qquad\mbox{if }a\not\in\pi\\
    1\qquad\mbox{if }a\in\pi
\end{cases}\forall a\in A^+
$$
$x_p\in\{0,1\}$ indicates whether an atom $p$ is achieved in a solution $\pi$.
$$x_p=
\begin{cases}
    0\qquad\mbox{if }p\not\in I(\pi)\\
    1\qquad\mbox{if }p\in I(\pi)
\end{cases}\forall p\in P
$$
$x_{a,p}\in\{0,1\}$ indicates whether $a$ is the first achiever of $p$.
$$x_{a,p}=
\begin{cases}
    1\qquad\mbox{if }a\mbox{ is the first achiever of }p\\
    0\qquad\mbox{otherwise}
\end{cases}\forall a\in A^+,\forall p\in add(a)
$$
Having defined the variables that will be used in our formulation, the model is defined as follows:
$$
\begin{cases}
    \displaystyle\min\sum_{ a\in A^+}x_a\cdot cost(a)\\
    \displaystyle_{(C1)}\quad\sum_{a\in A^+_p}x_{a,p}=x_p&\forall p\in P,\,A^+_p=\{a\in A^+,\,s.t.\,p\in add(a)\}\\
    \displaystyle_{(C2)}\quad\sum_{a\in A^+_{p,q}}x_{a,q}\leq x_p&\forall p,q\in P,\,A^+_{p,q}=\{a\in A^+,\,s.t.\,p\in pre(a),q\in add(a)\}\\
    _{(C3)}\quad x_{a,p}\leq x_a&\forall a\in A^+,\,p\in add(a)\\
    _{(C4)}\quad x_p=1&\forall p\in G
\end{cases}
$$

\begin{enumerate}
    \item[]$(C1)$: An atom is achieved iff an action is its first achiever; moreover there can be only one first achiever per atom;
    \item[]$(C2)$: An action can be a first achiever only if its preconditions are achieved;
    \item[]$(C3)$: An action can be a first achiever only if it's used;
    \item[]$(C4)$: Each atom in the goal state is achieved;
\end{enumerate}
Note that, since there's no upper bound on action indicator variables, 0-cost actions might be flagged as used, even if they aren't, and still be an optimal plan.\\
If we were just looking for the optimal cost $c^*$, this wouldn't bother us; however, since we are interested in the optimal plan $\pi^*$, we must acknowledge those actions by filtering the solution obtained by removing actions that aren't first achiever of any atom.

\section{Modeling acyclicity}
As shown both in \cite{Imai_15} and \cite{Rankooh_22}, any solution $\pi$ needs to be acyclic for it to be a plan, hence the need to add variables and constraints to force such acyclicity in the base model.\\
Multiple methods have been proposed: the time labeling method was first proposed in \cite{Imai_15} while vertex elimination was proposed in \cite{Rankooh_22}; the latter, however, proposed a revised time labeling method, proving that it was better than its predecessor, hence we will only mention the improved version.

\subsection{Time labeling}
For the time labeling method, we just need to create for each atom $p$, a timestamp $t_p\in\{0,...,|A^+|\}$.\\
The constraints to add to the model will then be:
$$_{(C5)}\quad t_p + 1 \leq t_q + |P|(1-x_{a, q})\qquad\forall a\in A^+,p\in pre(a),q\in add(a)$$
This constraint reads: if $a$ is first achiever of $q$, then every $p\in pre(a)$ must have a timestamp that precedes than $q$.

The correctness of this model is implied from its construction, having tackled directly the problem of having to construct a timeline of actions out of the obtained solution with constraint ${(C5)}$.

\subsection{Vertex elimination graph}
To explain how to use a vertex elimination graph to model acyclicity, we first need to define a \textit{causal relation graph} \cite{Rankooh_22_2}.
A causal relation graph of $\Pi^+$ is $G_{\Pi^+}=(P,E_{\Pi^+}),\mbox{ where }E_{\Pi^+}=\{(p,q),\,s.t.\,\exists a\in A^+\mbox{ with }p\in pre(a),q\in add(a)\}$.
As shown in \cite{Rankooh_22}, let $\mathcal{O}$ be an elimination ordering for $G_{\Pi^+}$, and $G^*_{\Pi^+}=(P,E^*_{\Pi^+})$ be the vertex elimination graph of $G_{\Pi^+}$ according to $\mathcal{O}$. Let $\Delta$ be the set of all triangles produced by elimination ordering $\mathcal{O}$ for graph $G_{\Pi^+}$: members of $\Delta$ are all ordered triples $(p,q,r)\,s.t.\,(p,r)$ is added to $E^*_{\Pi^+}$ by eliminating $q$.\\
We then create a variable for each edge $(p,q)\in E^*_{\Pi^+}$, $e_{p,q}\in\{0,1\}\,\forall (p,q)\in E^*_{\Pi^+}$ and add the following constraints:
$$
\begin{cases}
    _{(C6)}\quad x_{a,q}\leq e_{p,q}&\forall a\in A^+,p\in pre(a),q\in add(a)\\
    _{(C7)}\quad e_{p,q}+e_{q,p}\leq 1&\forall(p,q)\in E^*_{\Pi^+}\\
    _{(C8)}\quad e_{p,q}+e_{q,r}-1\leq e_{p,r}&\forall(p,q,r)\in\Delta
\end{cases}
$$
The correctness of the resulting model, composed of constraints $(C1)$ to $(C4)$ and $(C6)$ to $(C9)$, is proved in \cite{Rankooh_22}.

We will identify the two formulations respectively $\tl$ and $\ve$.

\section{Preprocessing}
\subsection{Landmark-based model reduction}
A \textit{landmark} is an element which needs to be used in every feasible solution \cite{Hoffman_04}. We define \textit{fact landmarks} and \textit{action landmarks}, as in \cite{Gefen_12}: a fact landmark of a planning task $\Pi$ is an atom that becomes true in some state of every plan, and similarly, an action landmark of a planning task $\Pi$ is an action that is included in every plan. Moreover, a fact or action landmark $l$ if a landmark for an atom $p$ if $l$ is a landmark for the task $(P,A,I,\{p\},cost)$, and similarly, $l$ is a landmark for an action $a$ if $l$ is a landmark for the task $(P,A,I,pre(a),cost)$.
For our delete-relaxed task $\Pi^+$, this leads to an obvious simplification of our model:
$$x_p,x_a\equiv1\qquad\forall p\in P,a\in A^+\mbox{ that is a fact/action landmark for an atom in }G$$

We can easily find a fact landmark $p$ or an action landmark $a$ by checking wether $(P,A^+\setminus A^+_p,I,G,cost)$, with $A^+_p=\{a\in A^+,\,s.t.\;p\in add(a)\}$, or $(P,A^+\setminus \{a\},I,G,cost)$ respectively, are infeasible tasks; however, since this is algorithm will be accompanied by other, more powerful, preprocessing algorithms, we opted for the efficient extraction method, which however doesn't compute all landmarks, proposed in \cite{Imai_15}:
\begin{algorithm}[h]
    \caption{Efficient landmark extracion algorithm}
    \begin{algorithmic}
        \State $L[p]\gets P\mbox{ for each } p\in P$ \qquad \# $L[p]$ stores candidates of fact landmarks for $p\in P$
        \State $S\gets\emptyset$
        \For{$a\in A^+$}
            \State $\mbox{insert }a\mbox{ into a FIFO queue }Q\mbox{ if }pre(a)\subseteq S$
        \EndFor
        \While{$Q\not=\emptyset$}
            \State retrieve an action $a$ from $Q$
            \For{$p\in add(a)$}
                \State $S\gets S\cup\{p\}$
                \State $X\gets L[p]\cap(add(a)\cup\bigcup_{p'\in pre(a)}L[p'])$
                \If{$L[p]\not= X$}
                    \State $L[p]\gets X$
                    \For{$a'\in\{a\in A^+,\,s.t.\;p\in pre(a)\}$}
                    \State insert $a'$ into $Q$ if $pre(a')\subseteq S$ and $a'\not\in Q$
                    \EndFor
                \EndIf
            \EndFor
        \EndWhile
        \State \# Now $L[p]$ contains the set of fact landmarks for $p\in P$
    \end{algorithmic}
\end{algorithm}

\subsection{First achievers filtering}

\subsection{Relevance analysys}

\subsection{Dominated actions extraction}

\subsection{Inverse actions extraction}

By applying preprocessing to $\tl$ and $\ve$, we obtain $\tle$ and $\vee$ respectively.