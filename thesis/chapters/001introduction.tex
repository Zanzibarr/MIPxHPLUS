AI Planning, a sub-area of Artificial Intelligence, has the objective of finding a sequence of actions, in a declaratively described system, to reach its goals while optimizing overall performance measures. Automated planners find the action to apply in each given state out of the set of possible actions for that state.\\
Fast Downward \cite{Helmert_06} is a classical planning system based on heuristic search, which is leading the state-of-the-art in classical planning. One such heuristics that can be used to efficiently explore the search space, is the \textit{delete-free relaxation} of the original problem, denoted as $\hplus$.

\section{Delete-free relaxation}
The value of $\hplus$ for a given planning problem is the optimal cost of the corresponding delete-relaxed planning problem, obtanied from the original problem by removing delete effects from all actions.\\
The value of $\hplus$ is a lower bound of the optimal cost of the original problem, making it an admissible heuristic, and has been shown that having $\hplus$ can significantly improve the efficiency of optimal planning \cite{Helmert_09}. Computing $\hplus$, however, is NP-equivalent \cite{Bylander_94} and is also hard to approximate \cite{Helmert_09}.

Methods for computing $\hplus$ can output optimal plans for a given delete-relaxed planning problem, equivalently important to find efficiently, since there exist delete-relaxed planning tasks that are of interest for the planning community. Moreover, optimal plans for classical planning problems, can be produced by iteratively solving and reformulating delete-relaxed tasks \cite{LB_Haslum_12}. This can be done by repeatedly finding optimal plans for a delete-relaxed version of the original planning problem, and reducing the relaxation by reformulating the problem whenever a plan has been found for the delete-relaxed task, that is not a valid plan for the original problem.

We will first investigate on preprocessing techniques proposed by \cite{Imai_15} and \cite{MLM_Haslum_12}, aiming to simplify the problem by taking advantage of the less restrictive structure of the delete-relaxed task, to then explore the use of primal heuristics, to provide a warm-start to our MIP solver, combining a greedy algorithm with concepts used to compute $\hadd$ \cite{Bonet_01}.\\
Finally we will implement the models proposed by \cite{Imai_15} and \cite{Rankooh_22}, which use different ways of modeling acyclicity in the model, respectively time labeling and vertex elimination, and compare them to our technique (of modeling acyclicity) of iteratively compiling violated disjunctive action landmarks \cite{MLM_Haslum_12}, \cite{Bonet_11} and subtour elimination constraints (S.E.C.), adding those to the base model as cuts.

\section{Preliminary definitions}
There are multiple description languages for planning tasks: STRIPS, ADL, SAS$^+$ and PDDL; we will lean on the STRIPS formalism \cite{Fikes_71}.

A \textit{STRIPS planning task} is a 5-tuple $\Pi=(P,A,I,G,cost)$: $P$ is a finite set of boolean variables (or \textit{atomic propositions, atoms} for short); $A$ is a finite set of \textit{actions}: each action is a triple $(pre(a), add(a), del(a))$, where $pre(a),add(a),del(a)\subseteq P$ denote the set of \textit{preconditions}, \textit{positive effects} and \textit{negative effects} of $a$; $I,G\subseteq P$ are respectively the \textit{initial state} and \textit{goal states}; $cost:A\to\mathbb{N}$ is a cost function that maps each action to a non-negative integer.

Executing an action $a$ from a state $S\subseteq P$ transforms the state into $S'\coloneq exec_a(S)\coloneq S\setminus del(a)\cup add(a)$.
Let $\pi=(a_0,...,a_n)$ be a sequence of actions and $S\subseteq P$ a state: we denote with $S(\pi)=exec_{a_n}(...(exec_{a_0}(S)))$, the state achieved through $\pi$, starting from $S$.
A \textit{solution} to a planning task is a sequence of actions that transform the system from the initial state to a state $I(\pi)\subseteq G$. Moreover a feasible solution, or \textit{plan}, is a solution $\pi=(a_0,...,a_n)$ that satisfies $\forall i,pre(a_i)\subseteq I((a_0,...,a_{i-1}))$, which means that each action's preconditions are achieved when using that action. Finally, the cost of an action sequence $\pi$ is $cost(\pi)=\sum_{a\in\pi}(cost(a))$, and our goal is to find an \textit{optimal plan} $\pi^*$ with minimum cost (\textit{optimal cost} $c^*$): $$\pi^*=argmin_{\pi}\{{cost(\pi)},\,s.t.\,\pi\mbox{ is a plan for }\Pi\},\;c^*=cost(\pi^*)$$
We will limit ourselves to the \textit{delete-relaxed task} $\Pi^+=(P,A^+,I,G,cost)$, where $A^+$ is obtained by replacing, from each action, the negative effects with the empty set. Moreover, we will limit to the case in which $I=\emptyset$ for simplicity: in a delete-relaxed task, an atom that is true in the initial state will remain true throughout the whole plan, making it useless. This won't restrict the use case of this work to instances with empty initial states, since it's easy to remove the initial state from any instance without changing its optimal plan: $\Pi^+_I=(P\setminus I, A^+_{I}, \emptyset, G\setminus I, cost)$, where $A^+_I=\{(pre(a)\setminus I, add(a)\setminus I, \emptyset),\forall a\in A^+\}$.

\section{Integer programming for classical planning}

TODO ...